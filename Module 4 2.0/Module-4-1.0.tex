\documentclass[]{article}
\usepackage{lmodern}
\usepackage{amssymb,amsmath}
\usepackage{ifxetex,ifluatex}
\usepackage{fixltx2e} % provides \textsubscript
\ifnum 0\ifxetex 1\fi\ifluatex 1\fi=0 % if pdftex
  \usepackage[T1]{fontenc}
  \usepackage[utf8]{inputenc}
\else % if luatex or xelatex
  \ifxetex
    \usepackage{mathspec}
  \else
    \usepackage{fontspec}
  \fi
  \defaultfontfeatures{Ligatures=TeX,Scale=MatchLowercase}
\fi
% use upquote if available, for straight quotes in verbatim environments
\IfFileExists{upquote.sty}{\usepackage{upquote}}{}
% use microtype if available
\IfFileExists{microtype.sty}{%
\usepackage[]{microtype}
\UseMicrotypeSet[protrusion]{basicmath} % disable protrusion for tt fonts
}{}
\PassOptionsToPackage{hyphens}{url} % url is loaded by hyperref
\usepackage[unicode=true]{hyperref}
\hypersetup{
            pdftitle={Module 4 - Wes Gordon},
            pdfborder={0 0 0},
            breaklinks=true}
\urlstyle{same}  % don't use monospace font for urls
\usepackage[margin=1in]{geometry}
\usepackage{graphicx,grffile}
\makeatletter
\def\maxwidth{\ifdim\Gin@nat@width>\linewidth\linewidth\else\Gin@nat@width\fi}
\def\maxheight{\ifdim\Gin@nat@height>\textheight\textheight\else\Gin@nat@height\fi}
\makeatother
% Scale images if necessary, so that they will not overflow the page
% margins by default, and it is still possible to overwrite the defaults
% using explicit options in \includegraphics[width, height, ...]{}
\setkeys{Gin}{width=\maxwidth,height=\maxheight,keepaspectratio}
\IfFileExists{parskip.sty}{%
\usepackage{parskip}
}{% else
\setlength{\parindent}{0pt}
\setlength{\parskip}{6pt plus 2pt minus 1pt}
}
\setlength{\emergencystretch}{3em}  % prevent overfull lines
\providecommand{\tightlist}{%
  \setlength{\itemsep}{0pt}\setlength{\parskip}{0pt}}
\setcounter{secnumdepth}{0}
% Redefines (sub)paragraphs to behave more like sections
\ifx\paragraph\undefined\else
\let\oldparagraph\paragraph
\renewcommand{\paragraph}[1]{\oldparagraph{#1}\mbox{}}
\fi
\ifx\subparagraph\undefined\else
\let\oldsubparagraph\subparagraph
\renewcommand{\subparagraph}[1]{\oldsubparagraph{#1}\mbox{}}
\fi

% set default figure placement to htbp
\makeatletter
\def\fps@figure{htbp}
\makeatother


\title{Module 4 - Wes Gordon}
\author{}
\date{\vspace{-2.5em}}

\begin{document}
\maketitle

\section{Puppy Fever}\label{puppy-fever}

\subsubsection{How many puppies can I
buy?}\label{how-many-puppies-can-i-buy}

I have puppy fever and I want to get as many puppies I can without
spending over \$1,000. Calculate how many puppies I can buy based on the
market price found online. Based on my calculations I can get 2 puppies
for \$700. I almost could get one more but am \$50 short.

\begin{verbatim}
Calculate how many puppies I can buy:

x <- 1000 #Budget Variable
y <- 200 #Puppy Price Variable

#The price for each puppy
puppy_price <- y 
print(puppy_price)

#Calculates the total cost of puppies
total_cost <- max_puppies * puppy_price 
print(total_cost)

#Logical test to show if purchase is too expensive
too_expensive <- total_cost > x 
print(too_expensive)
  
#Calculates the max amount of puppies I can get      
max_puppies <- floor( x / puppy_price) 
print(max_puppies)
\end{verbatim}

\section{Part 2}\label{part-2}

\section{ID Tags}\label{id-tags}

\subsubsection{ID tags for team}\label{id-tags-for-team}

I am making name tags for our team.

\begin{verbatim}
print(my_name <- c("Wesley Gordon"))
class(my_name <- c("Wesley Gordon"))

print(my_height <- "77 inches")
class(my_height <- "77 inches")

print(favorite_day <- as.Date("2025-07-04"))
class(favorite_day <- as.Date("2025-07-04"))

print(favorite_quote <- "Failure to prepare is preparing to fail")
class(favorite_quote <- "Failure to prepare is preparing to fail")
nchar("Failure to prepare is preparing to fail.")

#Class shows character despite my_height being numeric.
print(id <- c(my_name, my_height, favorite_day, favorite_quote))
class(id <- c(my_name, my_height, favorite_day, favorite_quote)) 

cat(id <- c(my_name, my_height, favorite_day, favorite_quote))
paste(id <- c(my_name, my_height, favorite_day, favorite_quote))

cat(paste(id <- c(my_name, my_height, favorite_day, favorite_quote)))
paste(cat(id <- c(my_name, my_height, favorite_day, favorite_quote)))

paste(id <- c(my_name, my_height, favorite_day, favorite_quote), collapse=', ')
paste(id <- c(my_name, my_height, favorite_day, favorite_quote, sep=', '))


cat(paste(my_name,my_height,favorite_day,favorite_quote, sep='_'))
\end{verbatim}

Try using cat and paste with id as a function argument. How do the
results differ? What happens when we use cat and paste at the same time
(i.e.~f(g(x)))? What happens if we change the order we use them
(i.e.~g(f(x)))? \#-CAT combines the vector into one line of characters
separated by spaces. PASTE changes the order by putting them in numbered
rows/columns. Each character string has quotations around them. Using
them at the same time, the first command directs the output. If CAT
precedes Paste it will execute CAT, and the inverse also applys.

How would you determine the difference between cat and paste using R
documentation (from within RStudio)? What is a great internet resource
to use as discussed in the book? \#-Right in R Console I can type the
command and formula help will display. A great internet resource is
Google searching ``how to do\_thing in R''. Also StackOverflow is a
robust Q\&A forum covering all things R programming.

What do sep and collapse arguments for paste do? If we wanted to append
each character variable in our vector id with a new line (i.e. ``\n'')
would we use sep or collapse? \#SEP from what I can see does nothing.
COLLAPSE can be used to be used to add commas after each word.

Display the contents of id using a combination of cat and paste with the
appropriate arguments for paste.
\#cat(paste(my\_name,my\_height,favorite\_day,favorite\_quote,
sep='\_'))

\end{document}
